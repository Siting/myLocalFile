\section{Problem 1: Data Transmission in Internet}

\subsection{part 1}
Advantages of Packet Switching v.s. Circuit Switching Transmission are:

\begin{itemize}
\item Packet switching is able to achieve higher utility in terms of bandwidth. The key feature of circuit switching is that it reserves a channel for a pair of hosts for a certain period of time, with and without actual data or messages being transmitted. In other words, there are times the channel sends nothing while keeping other hosts connect to each other.

However, packet switching allows packets between various hosts to use the same link during the same session. It increases the utility by allowing messages sent between one pair of hosts to take advantage of the dull time when the other pairs are not transmitting any messages.

\item Packet switching is more flexible when handling link failure situations compared to circuit switching. Packet switching is able to direct packets to other links when a link broke down.

\item Packet switching keeps sequence numbers for the transmitted packets which allows resending of specific packets if they are lost. However, circuit switching does not keep such sequence numbers and if a piece of message is lost, the entire file needs to be resend.
\end{itemize}

Disadvantages of packet switching are:

\begin{itemize}
\item Packet switching cannot guarantee a bandwidth while the circuit switching can. The delivery under packet switching is best-effort delivery since the link is not reserved for a specific pair of hosts, which means other hosts are competing for the bandwidth as well.

\item The forwarding in circuit switching would be much simpler compared to packet switching. Because a channel is reserved in circuit switching, routers visited by the transmitted messages forward messages to the same destination. Therefore, as long as a channel is alive, the routers can forward the messages to the same outgoing link without checking routing table.

However, in packet switching, when routers receive packets being assigned to different destinations, it requires the routers to identify which outgoing inks to send these packets. So more effort is made when forwarding packets in packet switching.
\end{itemize}

%%%%%%%%%%%%%%%%%%%%%%%%%%%%%%%%%%%%%%%%%%%%%%
\subsection{part 2}
\subsubsection{part a} NCP was not sufficient because it had no end-to-end host error control. To be more clear, NCP was designed only to serve ARPANET -- the only existing network which was so reliable that no error control was needed, which leads to the design that NCP had no end-to-end host error control. However, when network enlarged, network reliability became one critical issue which NCP could not accommodate. Therefore, NCP was not sufficient anymore.

The missing end-to-end host error control also restricts the size of Internet. When Internet dramatically grows, the reliability of Internet decreases which introduces corrupted packets. NCP does not provide the service to identify these corrupted packets. Another missing feature of NCP is congestion control. Since growing number of computers are trying to send data packets to the network, it is easy to forecast that there will be a high potential of traffic congestion. Therefore, having some protocol to restrict the amount of data being sent into the network is critical, which NCP did not have.
\subsubsection{part b} TCP/IP is a replacement protocol for NCP. The new features are:

\begin{enumerate}
\item TCP/IP provides reliable transmission by including a sequence number in its header. The sequence number is used to identify the order of received data packets and reconstruct the original message/file regardless of any disordering.
\item TCP/IP provides error detection since it has a checksum field in its header. After receiving a data packet, a checksum procedure will be performed to ensure the correctness of the received data.
\item TCP/IP provides sliding window flow control to avoid letting senders sending data too fast that receivers cannot properly receive and process. The flow control protocol lets receivers specify restrictions on the maximum amount of data a sender could send, and the sender could only move forward to sending more data after it received an acknowledgment from the receiver.
\item TCP/IP provides congestion control by adjusting the speed senders send data into the network. Generally speaking, if a potential traffic congestion is detected, senders decrease their sending rates to less the traffic goes into the network. If no sign of traffic congestion is detected, senders are allowed to increase their sending rates to take fully advantage of the network.
\end{enumerate}

%%%%%%%%%%%%%%%%%%%%%%%%%%%%%%%%%%%%%%%%%%%%%%%%%%%%
\subsection{part 3}

\begin{enumerate}
\item Domain Name System (DNS) was introduced to address the naming and addressing issue with the development of scale of Internet. The early stage network has a limited number of hosts and their names and addresses are able to be stored in a single table. However, as the number of hosts in network increases, the single table is not sufficient anymore. Therefore, the hierarchical distributed naming system, DNS, was developed to solve this issue by associating various information with domain names assigned to each of entities and it translates easily domain names to the numerical IP addresses to locate computers.
\item A hierarchical model of routing using an Interior Gateway Protocol (IGP) and an Exterior Gateway Protocol (EGP) was developed to connect the networks and regions together. Each region can run their selected IGP to deliver packets inside their region, while using EGP to route packets among different regions.
\end{enumerate}










